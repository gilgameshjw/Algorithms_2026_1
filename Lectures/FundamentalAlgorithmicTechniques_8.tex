
\documentclass{beamer}
\usepackage[utf8]{inputenc}
\usepackage{url}
\usepackage{algorithm}
\usepackage{algpseudocode}
\usepackage{lmodern}
\usepackage{natbib}
\usepackage{tikz}
%\documentclass
\usepackage{tikz}
\usetikzlibrary{graphs,graphdrawing}
\usegdlibrary{force}

\usepackage{physics}
\usepackage{graphicx} % Allows including images
\graphicspath{ {./images/} }
\usepackage{booktabs} % Allows the use of \toprule, \midrule and \bottomrule in tables
\usepackage{tikz}
\usetikzlibrary{arrows.meta}

\usetikzlibrary{graphs, graphdrawing}
\usegdlibrary{circular}




\usetikzlibrary{mindmap, trees, shadows, shapes, calc, fadings, positioning, decorations.pathreplacing, intersections, shapes, arrows}

%\usetheme{Hannover}
%\usecolortheme{spruce}
% EPIC FAIL
\usetheme{default}
%\usecolortheme{beetle}
\usepackage{graphics}

\usepackage{color}
\definecolor{new_turquoise}{RGB}{40,151,158}%{251,190,94}
\setbeamercolor{title}{fg=new_turquoise}


%\usepackage[colorlinks=true, urlcolor=blue, linkcolor=red]{hyperref}




\setbeamercolor{frametitle}{fg=new_turquoise}
\setbeamertemplate{itemize item}{\color{new_turquoise}$\blacksquare$}
\setbeamertemplate{itemize subitem}{\color{new_turquoise}$\blacksquare$}


% --- ENUMERATE ITEMS ---
\setbeamertemplate{enumerate item}{\color{new_turquoise}\insertenumlabel}
\setbeamertemplate{enumerate subitem}{\color{new_turquoise}\insertsubenumlabel}

\setbeamertemplate{caption}{\raggedright\insertcaption\par}

% --- COLOR THE TABLE OF CONTENTS ENTRIES ---
\setbeamercolor{section in toc}{fg=new_turquoise}
\setbeamercolor{subsection in toc}{fg=new_turquoise}
\setbeamercolor{subsubsection in toc}{fg=new_turquoise}



\usebackgroundtemplate{
    \includegraphics[width=\paperwidth,height=\paperheight]{figs/slide-title.jpg}
} 

\title{\fontsize{49}{7.2}{\bf Fundamental Algorithmic Techniques VIII}}
%\author{JW}
%\date{\color{new_turquoise}\today}
%S\titlegraphic{\includegraphics[width=2cm]{figs/jw.png}}






\begin{document}
\frame{\titlepage}
%% SLIDE 1 - INTRO TO THE TEAM
%%%%%%%%%%%%%%%%%%%%%%%%%%%%%%%%%%%%%%%%%%%%%%%%%%%%%

\usebackgroundtemplate{
    \includegraphics[width=\paperwidth,height=\paperheight]{figs/slide-pages}
} 


\setbeamertemplate{subsection in toc}{
  \color{new_turquoise}$\blacksquare$\color{black}~~\inserttocsubsection
}


% Outline frame
\begin{frame}{Outline}
    \tableofcontents
\end{frame}



\section{Search on Graphs}

\begin{frame}{Graph Traversals: BFS vs DFS}
\begin{columns}[T]
\column{0.5\textwidth}
\textbf{Breadth First Search}
\begin{itemize}
\item \textbf{Queue} (FIFO)
\item Level-order
\item Shortest path (unweighted)
\end{itemize}

\centering
\begin{tikzpicture}[baseline=(current bounding box.center)]
  \draw[fill=blue!10] (0,0) rectangle (3,0.6);
  \node at (0.5,0.3) {A}; \node at (1.5,0.3) {B}; \node at (2.5,0.3) {C};
  \draw[->, thick] (-0.5,0.3) -- (0,0.3); % enqueue
  \draw[->, thick] (3,0.3) -- (3.5,0.3);  % dequeue
  \node[below] at (1.5,0) {\footnotesize queue (FIFO)};
\end{tikzpicture}

\column{0.5\textwidth}
\textbf{Depth First Search}
\begin{itemize}
\item \textbf{Stack} (LIFO)
\item Deep-first, backtrack
\item Discovery/finish times
\item SCCs = Strongly Connected Components
\end{itemize}

\centering
\begin{tikzpicture}[baseline=(current bounding box.center)]
  \foreach \y/\lbl in {0/A, 0.6/B, 1.2/D}
    \draw[fill=green!10] (0,\y) rectangle (1.2,\y+0.6) node[midway] {\lbl};
  \draw[->, thick] (0.6,1.8) -- (0.6,1.85); % push
  \draw[->, thick] (0.6,0) -- (0.6,-0.05);   % pop
  \node[below] at (0.6,0) {\footnotesize stack (LIFO)};
\end{tikzpicture}
\end{columns}

\vspace{0.8em}
\centering
\footnotesize Both solve reachability — BFS: wide, DFS: deep
\end{frame}


\begin{frame}{Breadth First Search in Practice}
\begin{figure}
    \centering
    \includegraphics[width=0.8\textwidth]{Algos_figs/BFS.pdf} % Replace with your actual image file
    \caption{Example of breadth first search steps}
\end{figure}
\end{frame}



\begin{frame}{Analysis of Search}
    Search on graph: 
    $\mathcal{G} = (V, E),$
    \begin{itemize}
        \item Each edge uv in the component traversed twice \\
        $\Longrightarrow$ $2E + 1$
        \item Search in sparse! Adjacency matrix $\mathcal{O}(V)$, \\$\mathcal{O}(V^2)$ if not sparse!
    \end{itemize}
Time complexity: $\mathcal{O}(V + E )$
\end{frame}




\section{Advanced notions}


\begin{frame}{Cliques}
A \textbf{clique} is a subset of vertices in an undirected graph such that:
\begin{itemize}
    \item Every two distinct vertices are \textbf{adjacent}
    \item The induced subgraph is \textbf{complete}
\end{itemize}

\vspace{0.5em}
\textbf{Examples:}
\begin{columns}[T]
\column{0.5\textwidth}
\centering
\textbf{3-clique} (size 3)\\
\begin{tikzpicture}[scale=0.85, transform shape]
  \tikzset{vertex/.style = {circle, draw, minimum size=16pt, inner sep=0pt}}
  \node[vertex, fill=blue!20] (A) at (0,0) {A};
  \node[vertex, fill=blue!20] (B) at (1.5,0) {B};
  \node[vertex, fill=blue!20] (C) at (0.75,1.3) {C};
  \draw (A) -- (B) -- (C) -- (A);
\end{tikzpicture}

\column{0.5\textwidth}
\centering
\textbf{2-clique} (size 2)\\
\begin{tikzpicture}[scale=0.85, transform shape]
  \tikzset{vertex/.style = {circle, draw, minimum size=16pt, inner sep=0pt}}
  \node[vertex, fill=green!20] (X) at (0,0) {X};
  \node[vertex, fill=green!20] (Y) at (1.5,0) {Y};
  \draw (X) -- (Y);
  % Isolated node to show it's not part of the clique
  \node[vertex] (Z) at (0.75,1.2) {Z};
\end{tikzpicture}
\end{columns}

\vspace{0.8em}
\footnotesize
\textbf{Note:} 

\begin{itemize}
    \item Any single edge is a clique of size 2. The largest clique in a graph is the \textit{maximum clique} (NP-hard to compute).
    \item Bron-Kerbosch algorithm for finding maximum clique
\end{itemize}
\end{frame}


\begin{frame}{Minimum Spanning Tree (MST)}
A \textbf{spanning tree} of a connected, undirected graph $G = (V, E)$ is:
\begin{itemize}
\item A subgraph that is a \textbf{tree}
\item Includes \textbf{all vertices} ($|V|$ nodes)
\item Has exactly $|V| - 1$ edges (no cycles)
\end{itemize}

A \textbf{minimum spanning tree} (MST) is a spanning tree with the \textbf{smallest possible total edge weight}.

\vspace{0.5em}
\textbf{Example:}
\centering
\begin{tikzpicture}[scale=0.85, transform shape, >=stealth]
  \tikzset{
    vertex/.style = {circle, draw, minimum size=20pt, inner sep=0pt},
    edge/.style = {thick}
  }

  % Original graph (left)
  \node[vertex] (A) at (0,0) {A};
  \node[vertex] (B) at (2,1) {B};
  \node[vertex] (C) at (4,0) {C};
  \node[vertex] (D) at (2,-1) {D};

  % Edges with weights
  \draw[edge] (A) -- (B) node[midway, above] {1};
  \draw[edge] (A) -- (C) node[midway, below] {4};
  \draw[edge] (A) -- (D) node[midway, left] {3};
  \draw[edge] (B) -- (C) node[midway, above] {2};
  \draw[edge] (B) -- (D) node[midway, right] {5};
  \draw[edge] (C) -- (D) node[midway, below] {6};

  % MST (right)
  \node[vertex] (A2) at (6,0) {A};
  \node[vertex] (B2) at (8,1) {B};
  \node[vertex] (C2) at (10,0) {C};
  \node[vertex] (D2) at (8,-1) {D};

  % MST edges (thicker + blue)
  \draw[edge, blue, very thick] (A2) -- (B2) node[midway, above] {1};
  \draw[edge, blue, very thick] (B2) -- (C2) node[midway, above] {2};
  \draw[edge, blue, very thick] (A2) -- (D2) node[midway, left] {3};

  % Labels
  \node[below] at (2,-1.8) {Original graph};
  \node[below] at (8,-1.8) {MST (total weight = 6)};
\end{tikzpicture}

\vspace{0.5em}
\footnotesize
Used in network design, clustering, and approximation algorithms.
\end{frame}


\begin{frame}{Shortest Path}
\centering
\begin{tikzpicture}[scale=0.9, every node/.style={font=\footnotesize}, thick]
  % Vertices
  \foreach \pos/\name in {
    {(0,0)/s}, {(2,1.5)/A}, {(2,-1.5)/B}, {(4,0)/C},
    {(6,1.5)/D}, {(6,-1.5)/E}, {(8,0)/F}, {(10,0)/t}}
    \node[circle, draw, fill=white, minimum size=20pt] (\name) at \pos {$\name$};

  % Edges with weights (lightly grayed for background)
  \draw[gray!60] (s) -- (A) node[midway, above] {4};
  \draw[gray!60] (s) -- (B) node[midway, below] {2};
  \draw[gray!60] (A) -- (C) node[midway, right] {1};
  \draw[gray!60] (B) -- (C) node[midway, right] {3};
  \draw[gray!60] (A) -- (D) node[midway, above] {2};
  \draw[gray!60] (C) -- (D) node[midway, left] {5};
  \draw[gray!60] (C) -- (E) node[midway, right] {1};
  \draw[gray!60] (B) -- (E) node[midway, below] {6};
  \draw[gray!60] (D) -- (F) node[midway, above] {1};
  \draw[gray!60] (E) -- (F) node[midway, below] {2};
  \draw[gray!60] (F) -- (t) node[midway, above] {3};
  \draw[gray!60] (D) -- (E) node[midway, right] {4};

  % Shortest path: s → B → C → E → F → t
  % Weights: 2 + 3 + 1 + 2 + 3 = 11
  % (Verify: other paths like s-A-D-F-t = 4+2+1+3=10? Actually that's shorter! Let's correct.)

  % Let's redesign to ensure s→A→D→F→t is shortest:
  % s-A:4, A-D:2, D-F:1, F-t:3 → total = 10
  % s-B-C-E-F-t: 2+3+1+2+3 = 11 → not shortest

  % So highlight s → A → D → F → t
  \draw[blue, very thick] (s) -- (A);
  \draw[blue, very thick] (A) -- (D);
  \draw[blue, very thick] (D) -- (F);
  \draw[blue, very thick] (F) -- (t);

  % Optional: show total weight
  \node[blue, above] at (5,2.2) {total weight = 10};
\end{tikzpicture}

\vspace{1em}
\footnotesize
The \textbf{shortest path} from $s$ to $t$ minimizes the sum of edge weights.
\end{frame}

\begin{frame}{Graph Coloring}
A \textbf{proper coloring} assigns colors to vertices so that  
\textbf{no two adjacent vertices share the same color}.

\vspace{0.8em}
\begin{columns}[T]
\column{0.5\textwidth}
\centering
\textbf{Valid 3-coloring}\\
\begin{tikzpicture}[scale=0.85, vertex/.style={circle, draw, minimum size=20pt}]
  \node[vertex, fill=red!30] (A) at (0,0) {A};
  \node[vertex, fill=blue!30] (B) at (2,0) {B};
  \node[vertex, fill=green!30] (C) at (1,1.73) {C};
  \node[vertex, fill=red!30] (D) at (3,1.73) {D};
  
  \draw (A) -- (B) -- (C) -- (A);
  \draw (B) -- (D);
  \draw (C) -- (D);
\end{tikzpicture}

\column{0.5\textwidth}
\centering
\textbf{Invalid coloring}\\
\begin{tikzpicture}[scale=0.85, vertex/.style={circle, draw, minimum size=20pt}]
  \node[vertex, fill=red!30] (A) at (0,0) {A};
  \node[vertex, fill=red!30] (B) at (2,0) {B}; % ← same color as A, but adjacent!
  \node[vertex, fill=green!30] (C) at (1,1.73) {C};
  \node[vertex, fill=blue!30] (D) at (3,1.73) {D};
  
  \draw (A) -- (B) -- (C) -- (A);
  \draw (B) -- (D);
  \draw (C) -- (D);
  
  % Highlight conflict
  \draw[red, thick, <->] (A) -- (B) node[midway, below, black] {\footnotesize conflict};
\end{tikzpicture}
\end{columns}

\vspace{0.8em}
\footnotesize
The smallest number of colors needed is the \textbf{chromatic number}.
\end{frame}


\begin{frame}{Planar Graphs}
A graph is \textbf{planar} if it can be drawn in the plane  
\textbf{without edge crossings} (except at vertices).

\vspace{0.8em}
\begin{columns}[T]
\column{0.5\textwidth}
\centering
\textbf{Planar graph}\\
\begin{tikzpicture}[scale=0.8, every node/.style={circle, draw, minimum size=18pt, inner sep=0pt}]
  \node (A) at (0,0) {A};
  \node (B) at (2,0) {B};
  \node (C) at (2,2) {C};
  \node (D) at (0,2) {D};
  \node (E) at (1,1) {E};
  
  \draw (A) -- (B) -- (C) -- (D) -- (A);
  \draw (E) -- (A);
  \draw (E) -- (B);
  \draw (E) -- (C);
  \draw (E) -- (D);
\end{tikzpicture}
\smallskip

\footnotesize
Drawing without crossings → planar.

\column{0.5\textwidth}
\centering
\textbf{Non-planar graph}\\
\begin{tikzpicture}[scale=0.8, every node/.style={circle, draw, minimum size=18pt, inner sep=0pt}]
  % K_{3,3} bipartite non-planar
  \node (U1) at (0,1.5) {U};
  \node (U2) at (0,0.5) {V};
  \node (U3) at (0,-0.5) {W};
  
  \node (V1) at (2,1.5) {X};
  \node (V2) at (2,0.5) {Y};
  \node (V3) at (2,-0.5) {Z};
  
  \foreach \u in {U1,U2,U3}
    \foreach \v in {V1,V2,V3}
      \draw (\u) -- (\v);
      
  % Add a crossing indicator
  \draw[red, thick, <->] (0.9,1.0) -- (1.1,1.0); % node[midway, above, black, font=\tiny] {crossing}
\end{tikzpicture}
\smallskip

\footnotesize
$K_{3,3}$ (complete bipartite) is non-planar.
\end{columns}

%\vspace{0.6em}
%\footnotesize
%\textbf{Kuratowski’s Theorem}: A graph is non-planar iff it contains a subdivision of $K_5$ or $K_{3,3}$.
\end{frame}



\begin{frame}{Strongly Connected Components (SCCs)}
\small
  \textbf{Definition:} In a directed graph \(G = (V, E)\), a \textcolor{new_turquoise}{strongly connected component} is a maximal subset \(C \subseteq V\) such that for every pair \(u, v \in C\), there is a directed path from \(u\) to \(v\) \textbf{and} from \(v\) to \(u\).

  \vspace{0.5em}
  \textbf{Key ideas:}
  \begin{itemize}
  \item Every vertex belongs to exactly one SCC.
  \item SCCs partition the vertex set.
  \item The \textit{condensation} of \(G\) (contracting each SCC to a node) is a DAG.
  \end{itemize}

  \vspace{0.8em}
  \begin{columns}[T]
    \column{0.5\textwidth}
      \centering
      Original graph\\
      \begin{tikzpicture}[baseline, vertex/.style={circle, draw, minimum size=18pt, inner sep=0pt}, edge/.style={thick}]
        \node[vertex] (A) at (0,0) {A};
        \node[vertex] (B) at (1.5,0) {B};
        \node[vertex] (C) at (3,0) {C};
        \node[vertex] (D) at (1.5,-1.5) {D};
        \node[vertex] (E) at (3,-1.5) {E};
        
        \draw[edge] (A) -- (B);
        \draw[edge] (B) -- (C);
        \draw[edge] (C) to[bend left=20] (B);
        \draw[edge] (B) -- (D);
        \draw[edge] (D) -- (E);
        \draw[edge] (E) to[bend left=20] (D);
      \end{tikzpicture}

    \column{0.5\textwidth}
      \centering
      SCCs (condensation)\\
      \begin{tikzpicture}[baseline, vertex/.style={circle, draw, minimum size=18pt, inner sep=0pt}, edge/.style={thick}]
        \node[vertex, fill=blue!20] (AB) at (0,0) {A};
        \node[vertex, fill=red!20] (BC) at (1.8,0) {B,C};
        \node[vertex, fill=green!20] (DE) at (1.8,-1.5) {D,E};
        
        \draw[edge] (AB) -- (BC);
        \draw[edge] (BC) -- (DE);
      \end{tikzpicture}
  \end{columns}

  \vspace{0.5em}
  \textbf{Algorithms:} Kosaraju’s, Tarjan’s, or Gabow’s (all linear time: \(O(|V| + |E|)\)).
\end{frame}





\begin{frame}{Graph Transformations}
\begin{columns}
\begin{column}{0.33\textwidth}
\centering
\textbf{Original Graph}\\
\begin{tikzpicture}[scale=0.6]
\graph[circular placement, radius=1.5cm, nodes={circle, draw, minimum size=0.7cm}] {
    a, b, c, d;
    a -> {b, c};
    b -> {c, d};
    c -> {d};
};
\end{tikzpicture}
\end{column}

\begin{column}{0.33\textwidth}
\centering
\textbf{Transpose Graph}\\
(Edges reversed)\\
\begin{tikzpicture}[scale=0.6]
\graph[circular placement, radius=1.5cm, nodes={circle, draw, minimum size=0.7cm}] {
    a, b, c, d;
    b -> {a};
    c -> {a, b};
    d -> {b, c};
};
\end{tikzpicture}
\end{column}

\begin{column}{0.33\textwidth}
\centering
\textbf{Dual Graph}\\
(Faces become vertices)\\
\begin{tikzpicture}[scale=0.6]
\draw (0,0) -- (2,0) -- (2,2) -- (0,2) -- cycle;
\draw (0,0) -- (2,2);
\draw (2,0) -- (0,2);
\node[circle, draw, minimum size=0.5cm] at (1,1) {f1};
\node[circle, draw, minimum size=0.5cm] at (0.5,0.5) {f2};
\node[circle, draw, minimum size=0.5cm] at (1.5,0.5) {f3};
\node[circle, draw, minimum size=0.5cm] at (1.5,1.5) {f4};
\node[circle, draw, minimum size=0.5cm] at (0.5,1.5) {f5};
\end{tikzpicture}
\end{column}
\end{columns}

\begin{center}
\vspace{1em}
\textbf{Inverse Graph}: If $(u,v) \in E$, then $(u,v) \notin E_{inv}$ (complement)
\end{center}
\end{frame}







\end{document}


