\documentclass[11pt]{article}

% ======================
% Packages
% ======================
\usepackage[utf8]{inputenc}
\usepackage[T1]{fontenc}
\usepackage{lmodern}
\usepackage{amsmath, amssymb, amsthm}
\usepackage{geometry}
\usepackage{listings}
\usepackage{xcolor}
\usepackage{graphicx}
\usepackage{algorithm}
\usepackage{algpseudocode}
\usepackage{hyperref}
\usepackage{enumitem}

% ======================
% Page Layout
% ======================
\geometry{a4paper, margin=1in}
\setlength{\parindent}{0pt}
\setlength{\parskip}{1em}

% ======================
% Colors & Code Styling
% ======================
\definecolor{codegreen}{rgb}{0,0.6,0}
\definecolor{codegray}{rgb}{0.5,0.5,0.5}
\definecolor{codepurple}{rgb}{0.58,0,0.82}
\definecolor{backcolour}{rgb}{0.95,0.95,0.92}

\lstdefinestyle{mystyle}{
    backgroundcolor=\color{backcolour},
    commentstyle=\color{codegreen},
    keywordstyle=\color{magenta},
    numberstyle=\tiny\color{codegray},
    stringstyle=\color{codepurple},
    basicstyle=\ttfamily\small,
    breakatwhitespace=false,
    breaklines=true,
    captionpos=b,
    keepspaces=true,
    numbers=left,
    numbersep=5pt,
    showspaces=false,
    showstringspaces=false,
    showtabs=false,
    tabsize=2
}

\lstset{style=mystyle}

% ======================
% Theorem-like Environments
% ======================
\newtheorem{problem}{Problem}
\newtheorem{subproblem}{Subproblem}[problem]
\newtheorem{solution}{Solution}[problem]

% ======================
% Title Info
% ======================
\title{\textbf{Fundamental Algorithm Techniques} \\ Problem Set \#1}
%\author{Your Name \\ Student ID: 12345678}
\date{Due: January 27, 2026}

% ======================
% Document
% ======================
\begin{document}

\maketitle

\begin{problem}[Simplest Divide and Conquer on Sparse Vector]

  Consider a sparse vector of size n with only 0's and a single 1:
  \[
    v = [0,0,\dots,1,\dots,0]
  \]
  \begin{enumerate}[label=(\alph*)]
    \item Write pseudo recursive code that perform {\bf just} the binary divide and conquer method without creating a copy of the vector and runs until vector fully decomposed into sizes 1. 
    \begin{itemize}
        \item function divide(v , .., 2) 
        \item function divide(v, .., m) for any $2 \ll m$  
    \end{itemize}
    \item Analyse complexity of above divide and conquer: m=2, m=3,... tertiary division, which $T(n)$ and which $\mathcal{O}$?
    \item Next, once the division has reach sizes of 1, we collect the unique 1 and its position. 
    \begin{itemize}
        \item Evaluate the cost $f(n)$
        \item What is the recurrence relation $T(n)$
        \item is complexity now $\mathcal{O}(log(n))$ or $\mathcal{O}(n)$ (use master Theorem)
    \end{itemize}
    %\item  Compare with simpler approach: run over all indices...
  \end{enumerate}
\end{problem}


\begin{problem}[Multiplication in basis 10]
    Similar to Paesant multiplication, but in basis 10. it is just the first multiplication you learned at school.

    represent x, y in $\mathbb{N}^+$ with the vectors/arrays $X$ and $Y$, such that\\
    $x = \sum_{i=0}^{n_x} X[i]\cdot 10^i$, $y =\sum_{j=0}^{n_y} Y[j] \cdot 10^j$.

  \begin{enumerate}
      \item write a working multiplication code
      \item tweak code such that  results can be larger than the limit of your standard integer
      %\item what is the time complexity of your code?
      \item Explain how above multiplication can be described with the divide and conquer recursion to find $T(n)\approx4T(\frac{n}{2})$ or Karatsuba algorithm: $T(n)\approx3T(\frac{n}{2})$, finding with Master Theorem resp. $\mathcal{O}(n^2)$ and $\mathcal{O}(n^{1.585})$ {\bf hard} \\

    Hint 1: $x \cdot y = \left( x_1 \cdot 10^{n/2} + x_0 \right) \left( y_1 \cdot 10^{n/2} + y_0 \right)$ \\
    Hint 2: $x \cdot y = z_2 \cdot 10^{n} + z_1 \cdot 10^{n/2} + z_0$
      
      \item $\sum_{i=1}^n i$ or $n+ n-1 + \dots + 2 + 1$ can be computed by one application of school multiplication: $n! = \frac{1}{2} mult(v, w)$ with which v, w? {\bf hard } but simple math...
  \end{enumerate}
\end{problem}




\end{document}
